
%%%%%%%%%%%%%%%%%%%%%%% file typeinst.tex %%%%%%%%%%%%%%%%%%%%%%%%%
%
% This is the LaTeX source for the instructions to authors using
% the LaTeX document class 'llncs.cls' for contributions to
% the Lecture Notes in Computer Sciences series.
% http://www.springer.com/lncs       Springer Heidelberg 2006/05/04
%
% It may be used as a template for your own input - copy it
% to a new file with a new name and use it as the basis
% for your article.
%
% NB: the document class 'llncs' has its own and detailed documentation, see
% ftp://ftp.springer.de/data/pubftp/pub/tex/latex/llncs/latex2e/llncsdoc.pdf
%
%%%%%%%%%%%%%%%%%%%%%%%%%%%%%%%%%%%%%%%%%%%%%%%%%%%%%%%%%%%%%%%%%%%


\documentclass[runningheads,a4paper]{llncs}

\usepackage{amssymb}
\setcounter{tocdepth}{3}
\usepackage{graphicx}

\usepackage{url}
\urldef{\mailsa}\path|{alfred.hofmann, ursula.barth, ingrid.haas, frank.holzwarth,|
\urldef{\mailsb}\path|anna.kramer, leonie.kunz, christine.reiss, nicole.sator,|
\urldef{\mailsc}\path|erika.siebert-cole, peter.strasser, lncs}@springer.com|    
\newcommand{\keywords}[1]{\par\addvspace\baselineskip
\noindent\keywordname\enspace\ignorespaces#1}

\begin{document}

\mainmatter  % start of an individual contribution

% first the title is needed
\title{Modeling the Coevolutionary Dynamics of the \emph{Lobaria pulmonaria} Lichen Symbiosis}

% a short form should be given in case it is too long for the running head
\titlerunning{Coevolutionary Dynamics \emph{L. pulmonaria} Lichen Symbiosis}

% the name(s) of the author(s) follow(s) next
%
% NB: Chinese authors should write their first names(s) in front of
% their surnames. This ensures that the names appear correctly in
% the running heads and the author index.
%
\author{Simon Carrignon\textit{$^{1}$}, Salva Duran-Nebreda\textit{$^{2,}$}\textit{$^{3}$}, Aina Oll\'e-Vila\textit{$^{2,}$}\textit{$^{3}$}, Julia Adams\textit{$^{4}$}}
%


\authorrunning{Coevolutionary Dynamics \emph{L. pulmonaria} Lichen Symbiosis}
% (feature abused for this document to repeat the title also on left hand pages)

% the affiliations are given next; don't give your e-mail address
% unless you accept that it will be published
\institute{
\textit{$^{1}$~Barcelona Supercomputing Center, Carrer de Jordi Girona, 29-31, 08034 Barcelona, Spain.}\\
\textit{$^{2}$~Instituci\'o Catalana per a la Recerca i Estudis Avan\c{c}ats-Complex Systems Lab, Universitat Pompeu Fabra, 08003 Barcelona, Spain.}\\
\textit{$^{3}$~Institut de Biologia Evolutiva (CSIC-Universitat Pompeu Fabra), Passeig Mar\'itim de la Barceloneta 37, 08003 Barcelona, Spain.}\\
\textit{$^{4}$~Department of Botany and Plant Sciences, University of California at Riverside (UCR Lichen Herbarium), Riverside, CA 92521}}

%
% NB: a more complex sample for affiliations and the mapping to the
% corresponding authors can be found in the file "llncs.dem"
% (search for the string "\mainmatter" where a contribution starts).
% "llncs.dem" accompanies the document class "llncs.cls".
%

\toctitle{Coevolutionary Dynamics \emph{L. pulmonaria} Lichen Symbiosis}
\tocauthor{}
\maketitle

Introduction. 

A comprehensive understanding of symbiotic interactions will require quantitative analysis of the hyperdiverse and largely unexplored lichen symbioses between lichenized fungi (mycobionts mostly in phylum Ascomycota) and their photosynthetic hosts (i.e. photobionts), representing the best model of symbioses. The lichen symbiosis is an extremely successful nutritional mode in heterotrophic lichenized fungi, allowing the fungi to obtain carbohydrate-rich resources directly from their photobiont partner, which can either be cyanobacteria, green algae, or in some cases both [1]. In turn, the fungus may protect the photobiont from desiccation and reduce light intensity, allowing for adaptive radiation into new environments [2]. Although the nature of the lichen symbiosis is widely debated, many sources agree that the lichen system represents an ecologically obligate mutualistic interaction whereby the net fitness of all partners is maximized (i.e., all partners exhibit a maximum altitude on the fitness landscape in the mutualistic symbiosis) [3, 4, 5]. 

The symbiotic phenotype is only expressed when there is a compatible match between partners, resulting in the morphologically and physiologically integrated lichen thallus [4, 6]. Lichenization is an evolutionarily and ecologically successful strategy (> 20\% of fungi are lichenized), resulting in approximately 14,000  lichen species known to date [1, 5]. Interestingly, the mechanism and stages of lichen formation still remain unclear [1, 6]. All symbionts must be present, recognition must occur, and environmental conditions (extreme heat and cold stress in some cases) must be gathered together for successful developmental and thallus formation, making lichen reproduction complex [5, 6, 7]. Lichens can reproduce sexually via fungal spores (horizontal transmission) and asexually via vegetative propagules and thallus fragmentation (vertical transmission). In the sexual mode of reproduction, the fungal spores must interact with a compatible free-living algae and/or cyanobacterium in order to reconstitute the lichen thallus. In many lichen symbioses, the algal partner is not able to sexually reproduce [8]. In the asexual mode of reproduction, mycobionts and photobionts are co-dispersed via fragmentation of the main thallus body and specialized asexual propagules (isidia or soredia) (Fig. 1). Horizontal transmission of the green algae has also been shown to occur in nearly asexual lichen populations [9]. 

The mode of reproduction strongly influences the genetic structure of a lichen population [10, 11]. Due to the lack of genetic markers with high-marker resolution, few studies have characterized the within population genetic structure of lichen species (i.e. mycobiont and photobionts genotypes). Because Lobaria pulmonaria is the best-studied lichen species in the world, it is a good candidate for testing hypotheses related to the genetic structure of L. pulmonaria  populations. L. pulmonaria is a tripartite lichen species consisting of an association among the green alga S. reticulata, the cyanobacterium Nostoc sp., and its main fungal host in Europe, Lobaria [12, 13]. Recently, eight fungus-specific [14, 15, 10] and seven alga-specific [16] microsatellite markers have been developed for L. pulmonaria, allowing for reliable identification of genetically distinct individuals in highly clonal populations. To note, microsatellite markers have not been developed for the cyanobacterium [17]. 

To study the population structure of L. pulmonaria, the fungal and algal symbionts of 1960 L. pulmonaria thalli from 62 populations in forests throughout Europe, parts of North America, Asia, and Africa were genotyped at eight and seven microsatellite loci [10]. The L. pulmonaria-S. reticulata symbiosis showed significant within-population genetic structure due to restricted gene flow and vertical transmission (i.e. co-dispersal of vegetative propagules) with identical genotypes found in 77\% of fungal and 70\% of algal pairs [10]. L. pulmonaria was dominated by micro-evolutionary processes with high somatic mutation in the alga (30\%) and the fungus (15\%), while recombination contributed little to both the algal photobiont (no statistical evidence) and the fungal mycobiont (7.7\%) [10]. In the L. pulmonaria system, new genotypes continuously arise via mutation and selection [6]. If the nature of the fungal-algal partnership evolves over time, the lichen may be able to acquire properties that favor evolution in harsh and changing environments via the exploitation of new niches.

With the Dal Grande et al. (2012) dataset, we used different complex systems approaches such as network theory and agent-based models to better understand the structural genetic diversity existing within 62 lichen populations of L. pulmonaria and we reproduced general features of this system. First, we analyzed the fungal-algal population network structures of the data (8 fungal-specific and 7 algal-specific microsatellites) from the L. pulmonaria fungal-algal partnership. We then reconstructed the empirical bipartite genetic network, and obtained common network metrics (e.g., nestedness and modularity). Moreover, we introduced the particularities of the fungal-algal interactions in a continuous evolutionary algorithm based on the widely used ECHO framework [18; http://tuvalu.santafe.edu/projects/echo/]. With these analyses, we obtained similar patterns of diversification as well as ecological interactions, allowing us to better understand the mechanisms driving the evolution of symbionts in the L. pulmonaria system. Determining the co-evolutionary relationships and dynamics in the L. pulmonaria system will help us to better understand the role that symbiotic interactions play in the generation and maintenance of biodiversity in forest communities.

Methods. 

Empirical Network Analysis. 

We  used the Dal Grande et al. (2012) dataset to create a bipartite network of the relationships between the symbionts. We took the algae and the fungal lists of microsatellites separately. For each algae An (in symbiosis with a fungus Fn, we looked for the existence of another fungus Fm in symbiosis with a similar algae Am and thus created a link between An and Fm. As a preliminary test to check the validity of our methods, we considered two symbionts as similar when they shared at least six out of seven microsatellites (for the algae) and seven out of eight microsatellites for the fungi. The assumption behind this is that symbionts that are closer genetically have a higher probability to come from the same ancestor (Fig. 2). 

ECHO Model.

We constructed an agent-based model based on the widely used ECHO framework [18, 19]. The ECHO model typically consists of a collection of entities living in a simplified spatial domain, which can move around and interact with one another and with their environment. The interactions among agents can be used to model combat/confrontation, exchanges of goods or even mating in a more biological setting, and are driven by locality (co-localization of the entities) as well as agent-specific properties. These properties, usually called the agent genotype, are codified in the form of a symbol string or tag, and are used to determine the needs of the agents and the probabilities of interaction with the other entities through string matching [18] or, more generally, computing Hamming distances between agent?s genotypes. The ECHO model is also a continuous genetic algorithm  [20]; upon reproduction old genotypes are copied with slight mutations, giving rise to interesting evolutionary dynamics.

 In our case, we codified the different compartments of the L. pulmonaria system as well as the transitions between them (Fig 1). We used the tag system of ECHO to model the molecular recognition (receptors and physical embedding) between algae and fungi necessary to create the lichen. Each genotype consisted of a 11 bit string (thus allowing for $2^11=2048$ possible genotypes). Lichenization probabilities were calculated using the normalized similarity between bit strings (1-Hamming distance (G1,G2)/11). We considered two different lichenization functions, sigmoid (hill function with $n=2$) and michaelis-menten (saturation dynamics typical of enzymatic reactions). Additionally, other ecologically relevant features such as dispersal rates (introduced as random walks) and the ratio between sexual and asexual reproduction were included in the model. 
 
Simulations were initialized with a random uniform distribution of genotypes for both partners (algae and fungi) as well as randomized positions for the agents. After a fixed amount of iterations of the algorithm, snapshots of genotype composition of the population were taken (Fig. 4) and stored as a bipartite network. In particular, trying to reproduce the kind of information available from Dal Grande et al. (2012) we collected the genotypes of all the agents conforming a lichen for a given amount of algorithm iterations. Additionally, different kinds of ecological relations between algae and fungi were simulated: competition (both algae and fungi are better off on their own than forming a lichen), parasitism (only one type of agent benefits from the partnership) and mutualism (both agents benefit).

\textbf{Model Network Analysis.} 
For the bipartite networks retrieved from the ECHO model, we unraveled the modular structure of the lichen population to reveal the coevolution of the symbionts. We analyzed nestedness and modularity of the bipartite networks retrieved from the ECHO model and compared it to various null models via the BiMAT package in MATLAB [21]. For ecological networks (dealing with species? interactions), nestedness occurs when specialist species tend to interact with subsets of species that interact with more generalist species [22]. Although there is relative consensus on the meaning of nestedness, there are several distinct metrics by which it can be measured. The biMAT package implements the widely used NODF measure (0<=NODF<=1) of nestedness [23]. The modularity measure (0<=Qb<=1) we used detects communities including both types of nodes [24], as opposed to the measure where communities are formed by nodes of the same type [25]. 

\textbf{Results.} 

\textbf{Empirical Network Analysis. }

ECHO Model Network Analysis. 

From the different ecological interactions implemented in ECHO, no significant modular structure was found. However, the conditions of mutualism, with saturated and sigmoid lichenization functions showed statistically significant nestedness values. The z-score determines how different our measured value is from the distribution calculated from the randomized networks (null model), and the percentile tells us the percentage of random network distributions that have values inferior to the one obtained from our network. We did not retrieve any nestedness value higher than 0.5. We obtained a nestedness value of 0.18 (the maximum nestedness is 1). Although our value of 0.18 is significantly more nested than the null models, we cannot say definitively that it is a nested network. Regarding the modularity metric, no case was found to be statistically significant. The mean modularity of the randomized networks was practically equal to the one observed in our network. 

\section*{Acknowledgements}
This work was initiated at the 2016 Complex Systems Summer School (CSSS) at the Sante Fe Institute (SFI) and was supported by grants from the British Lichen Society and the British Ecological Society. The authors would like to thank all SFI resident faculty members, SFI external faculty members, SFI Omidyar Postdoctoral Fellows, and 2016 CSSS course participants for productive discussions as well as SFI staff during the 2016 CSSS. The authors would also like to thank Dr. Christoph Scheidegger for access to multiple datasets and Dr. Francesco Dal Grande for providing detailed knowledge of the L. pulmonaria system. 

\section*{References-falta cambiar a bibtex etc, cuando lo tengamos finiquitado mejor}
Lutzoni F,  Miadlikowska J. 2009. Lichens. Quick guide. Current Biology, 19:R502-R503.

Nash TH III. 2008. Introduction. In: Lichen Biology, 2nd edn (ed. Nash TH III), pp. 1-8. Cambridge University Press, Cambridge UK.

Bronstein JL. 1994. Our current understanding of mutualism. Quarterly Review of Biology, 69: 31-51

Ahmadjian V. 1993. The Lichen Symbiosis. John Wiley, New York city, New York.

Honneger R. 1998. The lichen symbiosis-what is so spectacular about it? Lichenologist, 30:193-212

Honneger R. 2008. Morphogenesis. In: Lichen Biology, 2nd edn (ed. Nash TH III), pp. 69-93. Cambridge University Press, Cambridge UK.

Stocker-Worgotter E, Hager A. 2008. Culture methods for lichens and lichen symbionts. In: Lichen Biology, 2nd edn (ed. Nash TH III), pp. 355-365. Cambridge University Press, Cambridge UK.

Grube M, Berg G, Andresson OS, Vilhelmsson O, Dyer PS, Miao PW. 2014. Lichen genomics-prospects and progress. In: The Ecological Genomics of Fungi, 1st edn (ed. Martin F), pp. 191-212. John Wiley \& Sons, Inc. 

Wornik S, Grube M. 2010. Joint dispersal does not imply maintenance of partnerships in lichen symbioses. Microbial Ecology, 59:150-157. 

Dal Grande F. 2012. Vertical and horizontal photobiont transmission within populations of a lichen symbiosis. Molecular Ecology, 21: 3159-3172.

Hill DJ. 2009. Asymmetric co-evolution in the lichen symbiosis caused by a limited capacity for adaptation in the photobiont. Botanical Review, 75: 326-338

Dal Grande, F. 2011. Phylogeny and co-phylogeography of a photobiont-mediated guild in the lichen family Lobariaceae. Dissertation. 

Skaloud P, Friedl T, Hallmann C, Beck A. 2016. Taxonomic revision and species delimitation of coccoid green algae currently assigned to the genus Dictyochloropsis (Trebouxiophyceae, Chlorophyte). Journal of Phycology, 1-19

Walser JC, Sperisen C, Soliva M, Scheidegger C. 2003. Fungus-specific microsatellite primers of lichens: application for the assessment of genetic variation on different spatial scales in Lobaria pulmonaria. Fungal Genetics and 
Biology, 40:72-82. 

Widmer I, Dal Grande F, Cornejo C, Scheidegger C. 2010. Highly varaiable microsatellite markers for the fungal and algal symbionts of the lichen Lobaria pulmonaria and challenges in developing biont-specific molecular markers for 
fungal associations. Fungal Biology, 114:538-544. 

Dal Grande F, Widmer I, Beck A, Scheidegger C. 2010. Microsatellite markers for Dictyochloropsis reticulata (Trebouxiophyceae), the symbiotic alga of the lichen Lobaria pulmonaria (L.). Conservation Genetics, 11:1147-1149. 

Christoph Scheidegger. 2016. Personal communication. 

Holland JH. 1995. Hidden order: How adaptation builds complexity. Basic Books

Holland JH. 1994. Echoing emergence: Objectives, rough definitions, and speculations for Echo--class models. in G.A. Cowan, D. Pines and D. Meltzer, eds. Complexity: Metaphors, Models and Reality. Addison--Wesley, Reading, MA.

Mitchell M. 1998. An introduction to genetic algorithms. MIT press

Flores C. 2016. BiMat: a MATLAB package to facilitate the analysis of bipartite networks. Methods in Ecology and Evolution, 7:127?132.  

Bascompte J, Jordano P. 2006. The structure of plant-animal mutualistic networks. In: Pascual, M. and Dunne, J. (eds.). Ecological networks. Oxford University Press, Oxford, US. Pages 143-159. 

Almeida-Neto M, Guimaraes P, Guimaraes PR Jr, Loyola RD, Ulrich W. 2008. A consistent metric for nestedness analysis in ecological systems: reconciling concept and measurement Oikos, 117: 1227 1239

Barber MJ. 2007. Modularity and community detection in bipartite networks. Physical Review E, 76: 066102

Larremore DB, Clauset A, Jacobs AZ. 2014. Efficiently inferring community structure in bipartite networks. Physical Review E, 90(1), 012805

\end{document}
