
A comprehensive understanding of symbiotic interactions will require quantitative analysis of the hyperdiverse and largely unexplored lichen symbioses between lichenized fungi (mycobionts mostly in phylum Ascomycota) and their photosynthetic hosts (i.e. photobionts), representing the best model of symbioses. The lichen symbiosis is an extremely successful nutritional mode in heterotrophic lichenized fungi, allowing the fungi to obtain carbohydrate-rich resources directly from their photobiont partner, which can either be cyanobacteria, green algae, or in some cases both [1]. In turn, the fungus may protect the photobiont from desiccation and reduce light intensity, allowing for adaptive radiation into new environments [2]. Although the nature of the lichen symbiosis is widely debated, many sources agree that the lichen system represents an ecologically obligate mutualistic interaction whereby the net fitness of all partners is maximized (i.e., all partners exhibit a maximum altitude on the fitness landscape in the mutualistic symbiosis) [3, 4, 5]. 

The symbiotic phenotype is only expressed when there is a compatible match between partners, resulting in the morphologically and physiologically integrated lichen thallus [4, 6]. Lichenization is an evolutionarily and ecologically successful strategy (> 20\% of fungi are lichenized), resulting in approximately 14,000  lichen species known to date [1, 5]. Interestingly, the mechanism and stages of lichen formation still remain unclear [1, 6]. All symbionts must be present, recognition must occur, and environmental conditions (extreme heat and cold stress in some cases) must be gathered together for successful developmental and thallus formation, making lichen reproduction complex [5, 6, 7]. Lichens can reproduce sexually via fungal spores (horizontal transmission) and asexually via vegetative propagules and thallus fragmentation (vertical transmission). In the sexual mode of reproduction, the fungal spores must interact with a compatible free-living algae and/or cyanobacterium in order to reconstitute the lichen thallus. In many lichen symbioses, the algal partner is not able to sexually reproduce [8]. In the asexual mode of reproduction, mycobionts and photobionts are co-dispersed via fragmentation of the main thallus body and specialized asexual propagules (isidia or soredia) (Fig. 1). Horizontal transmission of the green algae has also been shown to occur in nearly asexual lichen populations [9]. 

The mode of reproduction strongly influences the genetic structure of a lichen population [10, 11]. Due to the lack of genetic markers with high-marker resolution, few studies have characterized the within population genetic structure of lichen species (i.e. mycobiont and photobionts genotypes). Because Lobaria pulmonaria is the best-studied lichen species in the world, it is a good candidate for testing hypotheses related to the genetic structure of L. pulmonaria  populations. L. pulmonaria is a tripartite lichen species consisting of an association among the green alga S. reticulata, the cyanobacterium Nostoc sp., and its main fungal host in Europe, Lobaria [12, 13]. Recently, eight fungus-specific [14, 15, 10] and seven alga-specific [16] microsatellite markers have been developed for L. pulmonaria, allowing for reliable identification of genetically distinct individuals in highly clonal populations. To note, microsatellite markers have not been developed for the cyanobacterium [17]. 

To study the population structure of L. pulmonaria, the fungal and algal symbionts of 1960 L. pulmonaria thalli from 62 populations in forests throughout Europe, parts of North America, Asia, and Africa were genotyped at eight and seven microsatellite loci [10]. The L. pulmonaria-S. reticulata symbiosis showed significant within-population genetic structure due to restricted gene flow and vertical transmission (i.e. co-dispersal of vegetative propagules) with identical genotypes found in 77\% of fungal and 70\% of algal pairs [10]. L. pulmonaria was dominated by micro-evolutionary processes with high somatic mutation in the alga (30\%) and the fungus (15\%), while recombination contributed little to both the algal photobiont (no statistical evidence) and the fungal mycobiont (7.7\%) [10]. In the L. pulmonaria system, new genotypes continuously arise via mutation and selection [6]. If the nature of the fungal-algal partnership evolves over time, the lichen may be able to acquire properties that favor evolution in harsh and changing environments via the exploitation of new niches.

With the Dal Grande et al. (2012) dataset, we used different complex systems approaches such as network theory and agent-based models to better understand the structural genetic diversity existing within 62 lichen populations of L. pulmonaria and we reproduced general features of this system. First, we analyzed the fungal-algal population network structures of the data (8 fungal-specific and 7 algal-specific microsatellites) from the L. pulmonaria fungal-algal partnership. We then reconstructed the empirical bipartite genetic network, and obtained common network metrics (e.g., nestedness and modularity). Moreover, we introduced the particularities of the fungal-algal interactions in a continuous evolutionary algorithm based on the widely used ECHO framework [18; http://tuvalu.santafe.edu/projects/echo/]. With these analyses, we obtained similar patterns of diversification as well as ecological interactions, allowing us to better understand the mechanisms driving the evolution of symbionts in the L. pulmonaria system. Determining the co-evolutionary relationships and dynamics in the L. pulmonaria system will help us to better understand the role that symbiotic interactions play in the generation and maintenance of biodiversity in forest communities.


