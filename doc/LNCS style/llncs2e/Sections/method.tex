
\subsection{Empirical Network Analysis.}

We  used the \cite{dalgrande2012verticalandhorizontalphotobionttransmissionwithinpopulationsofalichensymbiosis} dataset to create a bipartite network of the relationships between the symbionts. We took the algae and the fungal lists of microsatellites separately. For each algae An (in symbiosis with a fungus Fn, we looked for the existence of another fungus Fm in symbiosis with a similar algae Am and thus created a link between An and Fm. As a preliminary test to check the validity of our methods, we considered two symbionts as similar when they shared at least six out of seven microsatellites (for the algae) and seven out of eight microsatellites for the fungi. The assumption behind this is that symbionts that are closer genetically have a higher probability to come from the same ancestor (Fig. 2). 


\subsection{ECHO Model.}

In order to better understand the interplay between the peculiar mode of reproduction associated to the {\em L. pulmonaria} system and the underlying population structure and evolutionary dynamics, we constructed an agent-based model based on the widely used ECHO framework [18, 19]. The ECHO model typically consists of a collection of entities living in a simplified spatial domain, which can move around and interact with one another and with their environment. The interactions among agents can be used to model different kinds of processes -such as mating-, and are driven by locality as well as agent-specific properties, namely the agents' genotypes. The ECHO model is also a continuous genetic algorithm  [20]; upon reproduction old genotypes are copied with slight mutations, giving rise to quantifiable evolutionary dynamics.

In our case, we used the tag system of ECHO to model the molecular recognition (receptors and physical embedding) between algae and fungi necessary to create the lichen. We considered two different lichenization functions based on similarity, sigmoid (hill function with $n=2$) and michaelis-menten (saturation dynamics). Additionally, other ecologically relevant features such as dispersal rates (introduced here as random walks) and the ratio between sexual and asexual reproduction were included in the model. Simulations were carried out assuming a wide range of ecological relations between the algae and fungi: competition (both algae and fungi are better off on their own than forming a lichen), parasitism (only one type of agent benefits from the partnership) and mutualism (both agents benefit).

\textbf{Model Network Analysis.} 
For the bipartite networks retrieved from the ECHO model, we unraveled the modular structure of the lichen population to reveal the coevolution of the symbionts. We analyzed nestedness and modularity of the bipartite networks retrieved from the ECHO model and compared it to various null models via the BiMAT package in MATLAB [21]. For ecological networks (dealing with species? interactions), nestedness occurs when specialist species tend to interact with subsets of species that interact with more generalist species [22]. Although there is relative consensus on the meaning of nestedness, there are several distinct metrics by which it can be measured. The biMAT package implements the widely used NODF measure (0<=NODF<=1) of nestedness [23]. The modularity measure (0<=Qb<=1) we used detects communities including both types of nodes [24], as opposed to the measure where communities are formed by nodes of the same type [25]. 


