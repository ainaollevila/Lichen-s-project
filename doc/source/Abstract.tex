
%%%%%%%%%%%%%%%%%%%%%%% file typeinst.tex %%%%%%%%%%%%%%%%%%%%%%%%%
%
% This is the LaTeX source for the instructions to authors using
% the LaTeX document class 'llncs.cls' for contributions to
% the Lecture Notes in Computer Sciences series.
% http://www.springer.com/lncs       Springer Heidelberg 2006/05/04
%
% It may be used as a template for your own input - copy it
% to a new file with a new name and use it as the basis
% for your article.
%
% NB: the document class 'llncs' has its own and detailed documentation, see
% ftp://ftp.springer.de/data/pubftp/pub/tex/latex/llncs/latex2e/llncsdoc.pdf
%
%%%%%%%%%%%%%%%%%%%%%%%%%%%%%%%%%%%%%%%%%%%%%%%%%%%%%%%%%%%%%%%%%%%


\documentclass[runningheads,a4paper]{llncs}

\usepackage{amssymb}
\setcounter{tocdepth}{3}
\usepackage{graphicx}

\usepackage{url}
\urldef{\mailsa}\path|{alfred.hofmann, ursula.barth, ingrid.haas, frank.holzwarth,|
\urldef{\mailsb}\path|anna.kramer, leonie.kunz, christine.reiss, nicole.sator,|
\urldef{\mailsc}\path|erika.siebert-cole, peter.strasser, lncs}@springer.com|    
\newcommand{\keywords}[1]{\par\addvspace\baselineskip
\noindent\keywordname\enspace\ignorespaces#1}

\begin{document}

\mainmatter  % start of an individual contribution

% first the title is needed
\title{Modeling the Coevolutionary Dynamics of the \emph{Lobaria pulmonaria} Lichen Symbiosis}

% a short form should be given in case it is too long for the running head
\titlerunning{Coevolutionary Dynamics \emph{L. pulmonaria} Lichen Symbiosis}

% the name(s) of the author(s) follow(s) next
%
% NB: Chinese authors should write their first names(s) in front of
% their surnames. This ensures that the names appear correctly in
% the running heads and the author index.
%
\author{Simon Carrignon\textit{$^{1,2}$}, Salva Duran-Nebreda\textit{$^{2,}$}\textit{$^{3}$}, Aina Oll\'e-Vila\textit{$^{2,}$}\textit{$^{3}$}, Julia Adams\textit{$^{4}$}}
%


\authorrunning{Coevolutionary Dynamics \emph{L. pulmonaria} Lichen Symbiosis}
% (feature abused for this document to repeat the title also on left hand pages)

% the affiliations are given next; don't give your e-mail address
% unless you accept that it will be published
\institute{
\textit{$^{1}$~Barcelona Supercomputing Center, Carrer de Jordi Girona, 29-31, 08034 Barcelona, Spain.}\\
\textit{$^{2}$~Instituci\'o Catalana per a la Recerca i Estudis Avan\c{c}ats-Complex Systems Lab, Universitat Pompeu Fabra, 08003 Barcelona, Spain.}\\
\textit{$^{3}$~Institut de Biologia Evolutiva (CSIC-Universitat Pompeu Fabra), Passeig Mar\'itim de la Barceloneta 37, 08003 Barcelona, Spain.}\\
\textit{$^{4}$~Department of Botany and Plant Sciences, University of California at Riverside (UCR Lichen Herbarium), Riverside, CA 92521}}

%
% NB: a more complex sample for affiliations and the mapping to the
% corresponding authors can be found in the file "llncs.dem"
% (search for the string "\mainmatter" where a contribution starts).
% "llncs.dem" accompanies the document class "llncs.cls".
%

\toctitle{Coevolutionary Dynamics \emph{L. pulmonaria} Lichen Symbiosis}
\tocauthor{}
\maketitle
\section*{Abstract}
Lichens are a hyperdiverse symbiotic group spread across the globe in widely different niches and climates. This partnership between a fungus (also called the mycobiont) and a photobiont (cyanobacteria, algae or both) allows the fungus to obtain carbohydrate-rich resources directly from their photosynthetic partner \cite{lutzoni2009lichens} while the fungus protects the photobiont from desiccation, leading to the coevolution of both species and adaptive radiation into new environments \cite{nash1996lichen}. Lichenization is an evolutionarily and ecologically successful strategy ($>$20\% of fungi are lichenized), resulting in approximately 14,000 lichen species known to date \cite{lutzoni2009lichens,honegger1998lichen}. Although the nature of the lichen symbiosis is still widely debated, many sources agree that the lichen system represents an ecologically obligate mutualistic interaction whereby the net fitness of all partners is maximized \cite{bronstein1994our,honegger1998lichen}.

%Lichens can reproduce sexually via fungal spores (horizontal transmission) and asexually via vegetative propagules and thallus fragmentation (vertical transmission). In the sexual mode of reproduction, the fungal spores must interact with a compatible free-living algae and/or cyanobacterium in order to reconstitute the lichen thallus. In the asexual mode of reproduction, mycobionts and photobionts are co-dispersed via fragmentation of the main thallus body and specialized asexual propagules.

Lichens can reproduce sexually and asexually. In the asexual mode of reproduction, mycobionts and photobionts are co-dispersed via fragmentation of the main thallus body or via specialized asexual propagules, resulting into a genetically identical lichen where the photobiont is vertically transmitted from the initial thallus to the new lichen. 
In the sexual mode of reproduction, the fungal spores are spread without the photobiont and have thus to find a compatible algae and/or cyanobacterium in order to reconstitute the lichen thallus (\emph{relichenization}). This mechanism leads to the horizontal transmission of a new photobiont from a neighbouring lichen or from a free-living photobiont, genetically different from the photobiont associated with the lichen from which the fungal spores come from. 


This peculiar mode of reproduction strongly influences the genetic structure of lichen populations \cite{dal2012vertical,dal2011phylogeny}, affecting dispersal and evolutionary rates. 
Though it has been recently shown, through the study of {\em L. pulmonaria} \cite{dal2012vertical}, that asexual reproduction could be the main way of reproduction for lichen, sexual reproduction is suspected to play a central role at larger evolutionary scale. Relichenization process could be a successful strategy for genetic recombination, allowing lichens to  explore wider ecological niche by sharing photobiont between different lichens already adapted to diverse local conditions. It as been hypothesized \cite{rikkinen2003ecological} that this process could lead to the emergence of photobiont-mediated guilds, where different species of lichen interact and exchange similar photobionts within an evolutionary coherent structure. These loosely integrated functional units, where different species can exchange chemical elements and genetics materials, are good candidates to explain the evolutionary success of lichens. They allow its different sub-components to evolve more or less independently, at different rates and under different ecological niche, thus making the whole community more responsive to environmental changes and able to adapt to a wider range of environmental conditions.

Nonetheless, studying the evolutionary dynamics of such assemblages is a difficult task. It involves a broad range of heterogeneous entities that co-evolve and interact ecologically at various spatial and temporal levels. This makes fields studies delicate and costly as they have to include lot of material from different places and different temporal scales. Experimental work is also difficult as lichens are fragile biological entities extremely dependent to their local environment, conditions complex and hard to reproduce in the laboratory.    

%This was evidenced by recent articles in the most studied lichen system: {\em L. pulmonaria}, where thalli from 62 populations in forests throughout Europe, North America, Asia, and Africa were genotyped at several hypervariable microsatellite loci \cite{dal2012vertical}. These studies concluded that the {\em L. pulmonaria-S. reticulata} symbiosis showed significant within-population genetic structure due to restricted gene flow and vertical transmission (i.e. co-dispersal of vegetative propagules). 
%
%The lichen symbiosis is extremely interesting, as the nature of the system is different from the widely studied plant-animal mutualistic systems \cite{rohr2014structural,bascompte2006asymmetric,bascompte2006structure,bastolla2009architecture,rezende2007non,guimaraes2011evolution}. Its different modes of reproduction, the fact that you find events of delichenization throughout evolution  \cite{lutzoni2001major} and that non of the partners is mobile (only throughout spores), make of it a rather particular system. Moreover, it has been conjectured the asymetry in the coevolution of the two species, due to a limited capacity of adaptation in the photobiont \cite{hill2009asymmetric} as well as that the relationship between the two species cannot always be considered a mutualism but rather a comensalism or even parasitism (with the fungi taking advantage over the photobiont)  \cite{ahmadjian1993lichen}. 

One solution to encompass such obstacles is the use of computer simulations. It allows the study of a wide range of parameters involving a massive number of heterogeneous entities. 
In order to do so and to unveil which are the potential factors driving the evolution of the lichen symbiosis and of broader ecological and evolutionary interactions, we constructed an agent-based model based on the widely used ECHO framework \cite{holland1999echoing,holland1995hidden}. The ECHO model typically consists of a collection of entities living in a simplified spatial domain, which can move around and interact with one another and with their environment. The interactions among agents can be used to model different kinds of processes -such as mating-, and are driven by locality as well as agent-specific properties, namely the agents' genotypes. The ECHO model is also a continuous genetic algorithm \cite{mitchell1998introduction}; upon reproduction old genotypes are copied with slight mutations, giving rise to quantifiable evolutionary dynamics.

In our case, we used the tag system of ECHO to model the molecular recognition (receptors and physical embedding) between algae and fungi necessary to create the lichen. We considered two different lichenization functions based on similarity, sigmoid (hill function with $n=2$) and Michaelis-Menten (saturation dynamics). Additionally, other ecologically relevant features such as dispersal rates (introduced here as random walks) and the ratio between sexual and asexual reproduction were included in the model. Simulations were carried out assuming a wide range of ecological relations between the algae and fungi: competition (both algae and fungi are better off on their own than forming a lichen), parasitism (only one type of agent benefits from the partnership) and mutualism (both agents benefit).

To legitimate the choice and the design of our model, we crossed the study of the simulations' results with empirical evidences. We first show the ability of our model to reproduce simple dynamics already well studied and understood. Then, we demonstrate that this ability can be scaled through a 'bottom-up' approach to different levels of increasing complexity; before generalising to encompass all the level of interaction, from simple well-known biological processes to broad interspecies co-evolutionary dynamics.

At the empirical level, we have first used a dataset form Dalgrande et al. 2012\cite{dal2012vertical} to understand the very local process of dispersion and reproduction of the lichen{\em L. pulmonaria}.
%, at the level of limited populations of the same lichen. 
Using tools borrowed from network theory, we show that our model can successfully reproduce the dynamics and the general properties exhibited by populations, implying that the implemented processes are consistent with the biological ones. In a second approach, we have studied a dataset from Dalgrande et al. 2014\cite{dal2014molecular}. In this case, some key features of the interactions between different species of lichens, thus coming from wider populations, are measured. We use our model to show that those key features can indeed emerge from the generalization and the expansion of the same process previously described, as far as certain particular conditions are met. 

In the future, and given the ability of our model to reproduce those first level of dynamics, we want to take further our simulations to see if the emergence of more complex entities is possible, and if it is the case, under which conditions. We expect that our results could help to measure the solidity of the photobiont-mediated guilds hypothesis, thus unveiling which are the factors driving the evolutionary dynamics of the lichen symbiosis.  


%The first consists of the fungal and algal symbionts of 1960 L. pulmonaria thalli from 62 populations in forests throughout Europs, America, Asia and Africa genotyped at eight and seven microsatellite loci, respectively. The sparseness of this dataset has only allowed us to find potential similarities at a sampling scale (for each region sample) with the population of lichens retrieved from the ECHO model. The second one, were seven alga-specific microsatellite of a particular algal species were genotyped for a total of 13 fungal species forming partnership, allowed us to compare the structure of the bipartite network arising from a higher level, where significant modular structure is found. Different measures from network theory have been used to unveil the structure of the studied bipartite networks, with different interesting results arising. Having the temporal scale in the ECHO model has allowed us to study the possible coevolution signatures that might be driving the evolution of the lichen symbiosis. 



%\section{Introduction}
%\label{sec:methods}
%Lichens are a hyperdiverse symbiotic group spread across the globe in widely different {\em niche} and climates. This partnership between fungi and a photobiont (cyanobacteria, algae or both) allows the fungi to obtain carbohydrate-rich resources directly from their photosynthetic partner [1] while the fungus protects the photobiont from desiccation, allowing for coevolution and adaptive radiation into new environments [2]. Lichenization is an evolutionarily and ecologically successful strategy ($>$20\% of fungi are lichenized), resulting in approximately 14,000 lichen species known to date [1, 5]. Although the nature of the lichen symbiosis is still widely debated, many sources agree that the lichen system represents an ecologically obligate mutualistic interaction whereby the net fitness of all partners is maximized [3, 4, 5].

Lichens can reproduce sexually via fungal spores (horizontal transmission) and asexually via vegetative propagules and thallus fragmentation (vertical transmission). In the sexual mode of reproduction, the fungal spores must interact with a compatible free-living algae and/or cyanobacterium in order to reconstitute the lichen thallus. In the asexual mode of reproduction, mycobionts and photobionts are co-dispersed via fragmentation of the main thallus body and specialized asexual propagules.

These peculiar mode of reproduction strongly influences the genetic structure of lichen populations [10, 11], affecting dispersal and evolutionary rates. This was evidenced by recent articles in the most studied lichen system: {\em L. pulmonaria}, where thalli from 62 populations in forests throughout Europe, North America, Asia, and Africa were genotyped at several hypervariable microsatellite loci [10]. These studies concluded that the {\em L. pulmonaria-S. reticulata} symbiosis showed significant within-population genetic structure due to restricted gene flow and vertical transmission (i.e. co-dispersal of vegetative propagules). 

The lichen symbiosis is extremely interesting, as the nature of the system is different from the widely studied plant-animal mutualistic systems. Its different modes of reproduction, the fact that you find events of delichenization throughout evolution, aposymbiotic partners, and the fact that are not mobile (only throughout spores), make of it a rather particular system. Moreover, it has been conjectured the asymetry in the coevolution of the two species, due to a limited capacity of adaptation in the photobiont[10] as well as the fact that the relationship between the two species cannot always be considered a mutualism but rather a comensalism or even parasitism (with the fungi taking advantage over the photobiont). 

In order to unveil which are the potential factors driving the evolution of the lichen symbiosis, as well as its population structure, we constructed an agent-based model based on the widely used ECHO framework [18, 19]. The ECHO model typically consists of a collection of entities living in a simplified spatial domain, which can move around and interact with one another and with their environment. The interactions among agents can be used to model different kinds of processes -such as mating-, and are driven by locality as well as agent-specific properties, namely the agents' genotypes. The ECHO model is also a continuous genetic algorithm  [20]; upon reproduction old genotypes are copied with slight mutations, giving rise to quantifiable evolutionary dynamics.

In our case, we used the tag system of ECHO to model the molecular recognition (receptors and physical embedding) between algae and fungi necessary to create the lichen. We considered two different lichenization functions based on similarity, sigmoid (hill function with $n=2$) and michaelis-menten (saturation dynamics). Additionally, other ecologically relevant features such as dispersal rates (introduced here as random walks) and the ratio between sexual and asexual reproduction were included in the model. Simulations were carried out assuming a wide range of ecological relations between the algae and fungi: competition (both algae and fungi are better off on their own than forming a lichen), parasitism (only one type of agent benefits from the partnership) and mutualism (both agents benefit).

On the other hand, we have studied two available data sets \cite{dalgrande2012verticalandhorizontalphotobionttransmissionwithinpopulationsofalichensymbiosis}[el otro].The first consists of the fungal and algal symbionts of 1960 L. pulmonaria thalli from 62 populations in forests throughout Europs, America, Asia and Africa genotyped at eight and seven microsatellite loci, respectively. The sparseness of this dataset has only allowed us to find potential similarities at a sampling scale (for each region sample) with the population of lichens retrieved from the ECHO model. The second one, were seven alga-specific microsatellite of a particular algal species were genotyped for a total of 13 fungal species forming partnership, allowed us to compare the structure of the bipartite network arising from a higher level, where significant modular structure is found. Different measures from network theory have been used to unveil the structure of the studied bipartite networks, with different interesting results arising. Having the temporal scale in the ECHO model has allowed us to study the possible coevolution signatures that might be driving the evolution of the lichen symbiosis. 


%With the same dataset (\cite{dalgrande2012verticalandhorizontalphotobionttransmissionwithinpopulationsofalichensymbiosis}), we used different complex systems approaches such as network theory and agent-based models to better understand the structural genetic diversity existing within the lichen populations as well as reproduce general features of this system. First, we analyzed the fungal-algal population network structures of the data (8 fungal-specific and 7 algal-specific microsatellites) from the fungal-algal partnership. We then reconstructed the empirical bipartite genetic network, and obtained common network metrics (e.g., nestedness and modularity). Moreover, we introduced the particularities of the fungal-algal interactions in a continuous evolutionary algorithm based on the widely used ECHO framework [18]. With these analyses, we obtained similar patterns of diversification as well as ecological interactions, allowing us to better understand the mechanisms driving the evolution of symbionts in the {\em L. pulmonaria} system. Determining the co-evolutionary relationships and dynamics in the this particular system will help us to better understand the role that symbiotic interactions play in the generation and maintenance of biodiversity in forest communities.

%We  used the \cite{dalgrande2012verticalandhorizontalphotobionttransmissionwithinpopulationsofalichensymbiosis} dataset to create a bipartite network of the relationships between the symbionts. 



%\section{Method}
%\label{sec:method}
%
\subsection{Empirical Network Analysis.}

We  used the \cite{dalgrande2012verticalandhorizontalphotobionttransmissionwithinpopulationsofalichensymbiosis} dataset to create a bipartite network of the relationships between the symbionts. We took the algae and the fungal lists of microsatellites separately. For each algae An (in symbiosis with a fungus Fn, we looked for the existence of another fungus Fm in symbiosis with a similar algae Am and thus created a link between An and Fm. As a preliminary test to check the validity of our methods, we considered two symbionts as similar when they shared at least six out of seven microsatellites (for the algae) and seven out of eight microsatellites for the fungi. The assumption behind this is that symbionts that are closer genetically have a higher probability to come from the same ancestor (Fig. 2). 


\subsection{ECHO Model.}

We constructed an agent-based model based on the widely used ECHO framework [18, 19]. The ECHO model typically consists of a collection of entities living in a simplified spatial domain, which can move around and interact with one another and with their environment. The interactions among agents can be used to model combat/confrontation, exchanges of goods or even mating in a more biological setting, and are driven by locality (co-localization of the entities) as well as agent-specific properties. These properties, usually called the agent genotype, are codified in the form of a symbol string or tag, and are used to determine the needs of the agents and the probabilities of interaction with the other entities through string matching [18] or, more generally, computing Hamming distances between agent?s genotypes. The ECHO model is also a continuous genetic algorithm  [20]; upon reproduction old genotypes are copied with slight mutations, giving rise to interesting evolutionary dynamics.

 In our case, we codified the different compartments of the L. pulmonaria system as well as the transitions between them (Fig 1). We used the tag system of ECHO to model the molecular recognition (receptors and physical embedding) between algae and fungi necessary to create the lichen. Each genotype consisted of a 11 bit string (thus allowing for $2^11=2048$ possible genotypes). Lichenization probabilities were calculated using the normalized similarity between bit strings (1-Hamming distance (G1,G2)/11). We considered two different lichenization functions, sigmoid (hill function with $n=2$) and michaelis-menten (saturation dynamics typical of enzymatic reactions). Additionally, other ecologically relevant features such as dispersal rates (introduced as random walks) and the ratio between sexual and asexual reproduction were included in the model. 
 
Simulations were initialized with a random uniform distribution of genotypes for both partners (algae and fungi) as well as randomized positions for the agents. After a fixed amount of iterations of the algorithm, snapshots of genotype composition of the population were taken (Fig. 4) and stored as a bipartite network. In particular, trying to reproduce the kind of information available from Dal Grande et al. (2012) we collected the genotypes of all the agents conforming a lichen for a given amount of algorithm iterations. Additionally, different kinds of ecological relations between algae and fungi were simulated: competition (both algae and fungi are better off on their own than forming a lichen), parasitism (only one type of agent benefits from the partnership) and mutualism (both agents benefit).

\textbf{Model Network Analysis.} 
For the bipartite networks retrieved from the ECHO model, we unraveled the modular structure of the lichen population to reveal the coevolution of the symbionts. We analyzed nestedness and modularity of the bipartite networks retrieved from the ECHO model and compared it to various null models via the BiMAT package in MATLAB [21]. For ecological networks (dealing with species? interactions), nestedness occurs when specialist species tend to interact with subsets of species that interact with more generalist species [22]. Although there is relative consensus on the meaning of nestedness, there are several distinct metrics by which it can be measured. The biMAT package implements the widely used NODF measure (0<=NODF<=1) of nestedness [23]. The modularity measure (0<=Qb<=1) we used detects communities including both types of nodes [24], as opposed to the measure where communities are formed by nodes of the same type [25]. 




%\section{Results}
%\label{sec:results}
%
\subsection{Empirical Network Analysis. }

\subsection{ECHO Model Network Analysis. }

From the different ecological interactions implemented in ECHO, no significant modular structure was found. However, the conditions of mutualism, with saturated and sigmoid lichenization functions showed statistically significant nestedness values. The z-score determines how different our measured value is from the distribution calculated from the randomized networks (null model), and the percentile tells us the percentage of random network distributions that have values inferior to the one obtained from our network. We did not retrieve any nestedness value higher than 0.5. We obtained a nestedness value of 0.18 (the maximum nestedness is 1). Although our value of 0.18 is significantly more nested than the null models, we cannot say definitively that it is a nested network. Regarding the modularity metric, no case was found to be statistically significant. The mean modularity of the randomized networks was practically equal to the one observed in our network. 




\section*{Acknowledgements}
This work was initiated at the 2016 Complex Systems Summer School (CSSS) at the Sante Fe Institute (SFI). The authors would like to thank all SFI resident faculty members, SFI external faculty members, SFI Omidyar Postdoctoral Fellows, and 2016 CSSS course participants for productive discussions as well as SFI staff during the 2016 CSSS. The authors would also like to thank Dr. Christoph Scheidegger for access to multiple datasets and Dr. Francesco Dal Grande for providing detailed knowledge of the \emph{L. pulmonaria} system. 
% O ponemos la source de dinero de todos o de nadie! no?
%This work was initiated at the 2016 Complex Systems Summer School (CSSS) at the Sante Fe Institute (SFI) and was supported by grants from the British Lichen Society and the British Ecological Society. The authors would like to thank all SFI resident faculty members, SFI external faculty members, SFI Omidyar Postdoctoral Fellows, and 2016 CSSS course participants for productive discussions as well as SFI staff during the 2016 CSSS. The authors would also like to thank Dr. Christoph Scheidegger for access to multiple datasets and Dr. Francesco Dal Grande for providing detailed knowledge of the \emph{L. pulmonaria} system. 

\bibliographystyle{splncs}
%\bibliographystyle{natbib}
\bibliography{bib/simon}
\end{document}
