Lichens are a hyperdiverse symbiotic group spread across the globe in widely different {\em niche} and climates. This partnership between fungi and a photobiont (cyanobacteria, algae or both) allows the fungi to obtain carbohydrate-rich resources directly from their photosynthetic partner [1] while the fungus protects the photobiont from desiccation, allowing for coevolution and adaptive radiation into new environments [2]. Lichenization is an evolutionarily and ecologically successful strategy ($>$20\% of fungi are lichenized), resulting in approximately 14,000 lichen species known to date [1, 5]. Although the nature of the lichen symbiosis is still widely debated, many sources agree that the lichen system represents an ecologically obligate mutualistic interaction whereby the net fitness of all partners is maximized [3, 4, 5].

Lichens can reproduce sexually via fungal spores (horizontal transmission) and asexually via vegetative propagules and thallus fragmentation (vertical transmission). In the sexual mode of reproduction, the fungal spores must interact with a compatible free-living algae and/or cyanobacterium in order to reconstitute the lichen thallus. In the asexual mode of reproduction, mycobionts and photobionts are co-dispersed via fragmentation of the main thallus body and specialized asexual propagules.

These peculiar mode of reproduction strongly influences the genetic structure of lichen populations [10, 11], affecting dispersal and evolutionary rates. This was evidenced by recent articles in the most studied lichen system: {\em L. pulmonaria}, where thalli from 62 populations in forests throughout Europe, North America, Asia, and Africa were genotyped at several hypervariable microsatellite loci [10]. These studies concluded that the {\em L. pulmonaria-S. reticulata} symbiosis showed significant within-population genetic structure due to restricted gene flow and vertical transmission (i.e. co-dispersal of vegetative propagules). 

With the same dataset (\cite{dalgrande2012verticalandhorizontalphotobionttransmissionwithinpopulationsofalichensymbiosis}), we used different complex systems approaches such as network theory and agent-based models to better understand the structural genetic diversity existing within the lichen populations as well as reproduce general features of this system. First, we analyzed the fungal-algal population network structures of the data (8 fungal-specific and 7 algal-specific microsatellites) from the fungal-algal partnership. We then reconstructed the empirical bipartite genetic network, and obtained common network metrics (e.g., nestedness and modularity). Moreover, we introduced the particularities of the fungal-algal interactions in a continuous evolutionary algorithm based on the widely used ECHO framework [18]. With these analyses, we obtained similar patterns of diversification as well as ecological interactions, allowing us to better understand the mechanisms driving the evolution of symbionts in the {\em L. pulmonaria} system. Determining the co-evolutionary relationships and dynamics in the this particular system will help us to better understand the role that symbiotic interactions play in the generation and maintenance of biodiversity in forest communities.


