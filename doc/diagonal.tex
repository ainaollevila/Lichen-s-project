Si tenemos una poblacion P24 con 7 lichenes asi :

List Lichen Algae -- Funghus (L_AF):

1 A_a -- F_a
2 A_b -- F_b
3 A_c -- F_c
4 A_d -- F_d
5 A_e -- F_e
6 A_f -- F_f
7 A_g -- F_g

Digamos que si miramos al nivel genetico : 
	Fb == Ff == Fg
	Ae == Ag

Entonces, en nuestro system tenemos 7 lichenes pero solamente 6 unicos genotipo de Algae (que llamaré AMLG_n) y 5 unicos genotipos de Fungi (FMLG_n ). Puedo escribir la listas de lichenes asi : 



1 A_a (AMLG_a) -- F_a (FMLG_a)   A_a (AMLG_a) -- F_a (FMLG_a)
2 A_b (AMLG_b) -- F_b (FMLG_b)   A_b (AMLG_b) -- F_b (FMLG_b)
3 A_c (AMLG_c) -- F_c (FMLG_c)   A_c (AMLG_c) -- F_c (FMLG_c)
4 A_d (AMLG_d) -- F_d (FMLG_d) = A_d (AMLG_d) -- F_d (FMLG_d) 
5 A_e (AMLG_e) -- F_e (FMLG_e)   A_e (AMLG_e) -- F_e (FMLG_e)
6 A_f (AMLG_f) -- F_f (FMLG_f)   A_f (AMLG_f) -- F_f (FMLG_b)
7 A_g (AMLG_g) -- F_g (FMLG_g)   A_g (AMLG_a) -- F_g (FMLG_b)

Despues ago la matriz de coocurrencia donde cuento cuantas veces un genotipo de un typo interacta con un genotipo de otro. 
Para hacer eso saco los genotipos que son similares. Me sale dos listas de genotipos unicos : 

List de Genotipocs de Algae (L_AMLG) : 

AMLG_a
AMLG_b
AMLG_c
AMLG_d
AMLG_e
AMLG_f

List Genotypos de Fungus (L_FMLG) : 

FMLG_a
FMLG_b
FMLG_c
FMLG_d
FMLG_e


La matriz que queremos hacer es L_AMG x L_FMG :

	FMLG_a FMLG_b FMLG_c FMLG_d FMLG_e
AMLG_a
AMLG_b
AMLG_c
AMLG_d
AMLG_e
AMLG_f
AMLG_g


Como los genotypos en la matriz estan "sorted" de la misma manera que los "lichenes" en la lista inicial (L_AF), la mayor parte del tiempo el primero genotipo de algae y el primero genotipo de Fungu correspondent a los genotypos encontrados en el premiero Lichen. Y entonces es normal que hay algo en este.

En este caso la matriz final estara :

	FMLG_a	FMLG_b	FMLG_c	FMLG_d	FMLG_e
AMLG_a	1	1	0	0	0	
AMLG_b	0	1	0	0	0
AMLG_c	0	0	1	0	0
AMLG_d	0	0	0	1	0
AMLG_e	0	0	0	0	1
AMLG_f	0	1	0	0	0


Parece que hay una diagonal pero sale del hecho que el orden usado entre los genotypos y los lichenes e lo mismo.
